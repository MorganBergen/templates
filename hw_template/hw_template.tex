\documentclass[11pt]{article}
% add optional 'showframe' to show where margins are
\usepackage[margin=1in, left=0.75in, right=0.75in]{geometry}

% Text package for boldface small caps
\usepackage{bold-extra}
% Use in preamble environment
\usepackage{changepage}
\usepackage{calc}
\usepackage{etoolbox}
\usepackage{fancyhdr}
\usepackage{pgffor}
\usepackage{tcolorbox}
\tcbuselibrary{skins}
\usepackage{xcolor}
\usepackage{xfp}
\usepackage{xparse}
% Packages for homework
\usepackage{amsmath}
\usepackage{cancel}
\usepackage{physics}
\usepackage{siunitx}
% allow use of \qty by siunitx, not physics
\AtBeginDocument{\RenewCommandCopy\qty\SI}

% Customize footer (it is set in hwTitle command)
\pagestyle{fancy}
\setlength\headheight{14pt}
% Remove header horizontal bar
\renewcommand{\headrulewidth}{0pt}

% counts the depth that we're currently on
\newcounter{counter}
% counters for each depth (total of 3)
\newcounter{counter1}
\newcounter{counter2}[counter1]
\newcounter{counter3}[counter2]

% Symbols to use for each depth
% e.g. Roman == I, II, III | roman == i, ii, iii
%      Alph  == A,  B,  C  | alph  == a,  b,  c
%     arabic == 1,  2,  3
% Also fnsymbol: https://www.overleaf.com/learn/latex/Counters#\fnsymbol{somecounter}
\csdef{symbol1}{\Roman}
\csdef{symbol2}{\Alph}
\csdef{symbol3}{\arabic}
% Title name: e.g. question, problem, part, etc.
\csdef{title1}{Question}
\csdef{title2}{Part}
\csdef{title3}{Section}


\newenvironment{question}[1][]{
    % indentation for the environment
    \begin{adjustwidth}{\inteval{3*\value{counter}}em}{}

    % \counter is how deep in we are
    \stepcounter{counter}
    \def\counter{counter\the\numexpr\value{counter}}
    % increase by one because we start at counter1, then counter2, etc
    % where e.g. counter1 is the counter for the first depth
    \stepcounter{\counter}
    % which symbol type do we use, e.g. 1, A, a, i and so on
    \def\symbol{symbol\the\numexpr\value{counter}}
    
    \ifstrempty{#1}{
        \def\questionTitle{\csuse{title\arabic{counter}} \csuse{\symbol}{\counter}}
    }{
        % Allow for custom choice of question title
        \def\questionTitle{#1}
    }

    % padding above title
    \vspace{2em}
    % title itself
    \begin{tcolorbox}[colback=brown!\inteval{70 - 20*\value{counter}}!white, coltitle=black]
        \underline{\begin{huge}\textbf{\textsc{\questionTitle}}\end{huge}}
    \end{tcolorbox}
    \noindent
}{
    \end{adjustwidth}
    \addtocounter{counter}{-1}
    \par
}

% reset the counters
\newcommand\resetCounters{
    \foreach\index in {1, ..., 6} {
        \ifcsname c@counter\index\endcsname
            \setcounter{counter\index}{0}
        \else\fi
    }
}

% Title for the homework with arguments: homework_title, name, date
\newcommand\hwtitle[3]{
    % Set footer in this command to grab the title and author
    \fancyfoot[L]{\footnotesize\textsc{#1 | #2}}
    \fancyfoot[C]{}
    \fancyfoot[R]{\footnotesize\thepage}
    \vspace{3em}
    \begin{center}
        \huge\textbf{#1}\vspace{0.8em}\par
        \Large #2\vspace{1em}\par
        #3\vspace{1.2em}
        \hspace*{\fill}\textcolor{black!20!white}{\rule[3px]{\textwidth}{3px}}\vspace{-1em}
    \end{center}
}

% Labeling question from the book
\NewDocumentCommand\book{O{}mg}{
    \vspace{-4.1em}
    \begin{adjustwidth}{}{1em}
        \begin{flushright}
            \footnotesize(\textit{From }\IfNoValueTF{#3}{
                    \textit{the book}
                }{#1#3#1}\textit{, \textbf{question #2}})    
        \end{flushright}
    \end{adjustwidth}   
    \par\vspace{2.5em}\noindent
}
% 'and' for inbetween two equations in align environment or $$
\newcommand\andeqs[1][and]{\qquad\text{#1}\qquad}

% space for tensor indices
\newcommand\none{\phantom{\mu}}

\newcommand\questiontext[1]{
    \begin{adjustwidth}{0.8in}{0.8in}
        \begin{tcolorbox}[empty, borderline south={1pt}{0pt}{black!30!white}]
            \small#1
        \end{tcolorbox}
    \end{adjustwidth}
}


\begin{document}
\hwtitle{Homework $N$}{First Last}{April 1, 2023}

\begin{question}
    Use the \verb|question| environment for the questions.
    
    \begin{question}
        They can be nested. By default, it's up to three levels, but that can be increased by adding more counters, symbols and titles shown in the preamble.
    \end{question}
    \begin{question}
        The questions will automatically update their numbering/lettering. This choice can also be changed via the symbols in the preamble.
        \begin{question}
            One subsubquestion
        \end{question}
        \begin{question}
            And another
            
            \begin{align}
                \nabla F=0
            \end{align}
        \end{question}
    \end{question}
\end{question}

\begin{question}
    \questiontext{Use \texttt{\char`\\ questiontext} to write out the question itself in this format. It uses a tcolorbox to create the spacing/border.}
    The values for the numberings/letterings will reset properly so no need to deal with that.
\end{question}

\begin{question}
    \book{13.37}{Super Good Book}
    Use the command \verb|\book| to reference a specific problem in a book.
\end{question}

\begin{question}[Custom Question 1]
    Use the optional argument to specify the name of the question.
    \begin{question}
        The counting/lettering will not be interrupted by custom question titles.
    \end{question}
    \begin{question}[Custom Part - $\int_a^bf(x)\dd{x}$]
        Any question can be customized.
    \end{question}
\end{question}

\resetCounters
\begin{question}
    The counters can be reset to 1 at any point with the command \verb|\resetCounters|.
\end{question}
\end{document}